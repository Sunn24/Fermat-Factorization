\documentclass[11pt,aspectratio=43,xcolor={dvipsnames},notheorems]{beamer}

%%%%%%%Begin thai template%%%%%%%
\usepackage[no-math]{fontspec}
\usepackage{xunicode}
\usepackage{xltxtra}
\XeTeXlinebreaklocale "th"
\XeTeXlinebreakskip = 0pt plus 0.5pt
\defaultfontfeatures{Mapping=tex-text} 
\newfontfamily{\thaifont}[Path = ./Header/Fonts/,BoldFont={THSarabunNew Bold.ttf},ItalicFont={THSarabunNew Italic.ttf},BoldItalicFont={THSarabunNew BoldItalic.ttf},Scale=MatchLowercase]{THSarabunNew.ttf}
\newfontfamily{\englishfont}[Path = ./Header/Fonts/,BoldFont={cmunbx.ttf},ItalicFont={cmunti.ttf},BoldItalicFont={cmunbi.ttf},Scale=MatchLowercase]{cmunrm.ttf}
\usepackage[Latin,Thai]{ucharclasses}
\setTransitionTo{Thai}{\thaifont}
\setTransitionFrom{Thai}{\englishfont}
%%%%%%%end thai template%%%%%%%
\usepackage{setspace}
\setstretch{1.1}
% \setlength{\parindent}{2em}
%\setlength{\emergencystretch}{2em}
%สำหรับการใช้ฟ้อนThai-Eng%

\usetheme{PaloAlto}
%\usecolortheme{orchid}
\usefonttheme{serif}
 

% \definecolor{msl}{RGB}{0,119,182}
\definecolor{msl}{HTML}{E48F45}
\definecolor{light}{HTML}{F5CCA0}
% \definecolor{msl}{HTML}{F0997D}
% \definecolor{light}{HTML}{FFC3A1}

\definecolor{blueblack1}{HTML}{003366}
\definecolor{blueblack2}{RGB}{0,0,80}
% \usecolortheme[RGB={179,157,219}]{structure}
\usecolortheme[named=msl]{structure}

% \setbeamersize{sidebar width left=2cm}
\setbeamercolor{frametitle}{bg=msl!20!msl}
\setbeamercolor{sidebar}{bg=msl}
\setbeamercolor{title in sidebar}{fg=white}
\colorlet{titleright}{light}
\colorlet{titlemid}{msl}
\colorlet{titleleft}{msl}


%% (Main) https://tex.stackexchange.com/questions/300533/shading-a-frame-title-in-paloalto-beamer
%% https://tex.stackexchange.com/questions/226691/how-can-i-have-a-horizontally-shaded-frametitle-in-beamer
%% https://tex.stackexchange.com/questions/174545/how-can-i-have-a-horizontally-shaded-block-title-in-beamer
%% https://tex.stackexchange.com/questions/309945/horizontally-shaded-title-in-beamer

%%% ขนาด sidebar ด้านซ้าย %%% 
% \makeatletter
% \setlength\beamer@sidebarwidth{1.5cm}
% \makeatother

%%% กรอบหน้า Title %%% 
\setbeamertemplate{title page}[default][colsep=-4bp,rounded=true,shadow=true]

%%% กรอบหน้าอื่น ๆ %%% 
\setbeamertemplate{blocks}[rounded][shadow=true]

\makeatletter

\pgfdeclarehorizontalshading[titleleft,titleright]{beamer@frametitleshade}{\paperheight}{%
    color(\beamer@sidebarwidth)=(titlemid);%
    color(\paperwidth)=(titleright)%
}

\pgfdeclareverticalshading[titleleft,titleright]{beamer@sidebar}{\beamer@sidebarwidth}{%
    color(0pt)=(titleright);%
    color(\sidebarheight)=(titleleft)%
}

%%% Shade in Sidebar ด้านบน %%%
\setbeamertemplate{headline}{%
    \begin{beamercolorbox}[wd=\paperwidth]{frametitle}%
        \ifx\beamer@sidebarside\beamer@lefttext%
        \else%
        \hfill%
        \fi%
        \ifdim\beamer@sidebarwidth>0pt%  
        \usebeamercolor[bg]{logo}%
        \begin{pgfpicture}{0pt}{0pt}{\paperwidth}{\beamer@headheight}%
            \usebeamercolor{frametitle right}%
            \pgfpathrectangle{\pgfpointorigin}{\pgfpoint{\paperwidth}{\beamer@headheight}}%
            \pgfusepath{clip}%
            \pgftext[left,base]{\pgfuseshading{beamer@frametitleshade}}%
        \end{pgfpicture}%
        \vskip-\beamer@headheight%
        \vrule width\beamer@sidebarwidth height \beamer@headheight%
        \hskip-\beamer@sidebarwidth%
        \hbox to \beamer@sidebarwidth{\hss\vbox to
            \beamer@headheight{\vss\hbox{\color{fg}\insertlogo}\vss}\hss}%
        \else%
        \vrule width0pt height \beamer@headheight%  
        \fi%
    \end{beamercolorbox}%
}

%%% Shade in Sidebar ด้านซ้าย %%%
\setbeamertemplate{sidebar canvas left}{%
    \pgfuseshading{beamer@sidebar}%
}

\makeatother


\usepackage{tikz}
\usepackage{pgfplots}
\pgfplotsset{compat=newest}
\usetikzlibrary{arrows,angles,quotes}
\usepackage{caption}
\usepackage{subcaption}
\usepackage{graphicx}
\graphicspath{{.}}
\setbeamercovered{transparent=5}
\usepackage{multicol}
\addtobeamertemplate{navigation symbols}{}{%
    \usebeamerfont{footline}%
    \usebeamercolor[fg]{footline}%
    \hspace{1em}%
    \insertframenumber/\inserttotalframenumber
}
%\logo{\includegraphics[height=1cm]{logo2.png}}
\usepackage{hyperref}
\hypersetup{
    colorlinks=true,
    linkcolor=black,
    filecolor=magenta,      
    urlcolor=blue
    }

% \usepackage{bibentry}
\usepackage{booktabs} % Allows the use of \toprule, 


\title{The Fermat-Kraitcheik Factorization Method}
% \subtitle{for SCMA 490 Seminar}
\author[Krittapas N.]{Krittapas Ngammuengman 6305146}
\date{\today}

% math mode package %
\usepackage{amsfonts}
\usepackage{amsmath}
\usepackage{mathtools}
\usepackage{amssymb}

% theoremstyle %
\usepackage{amsthm}
\theoremstyle{definition}
\newtheorem{definition}{บทนิยาม}[]

\theoremstyle{plain}
\newtheorem{theorem}{Theorem}[]
\newtheorem{lemma}[theorem]{บทตั้ง}
\newtheorem{proposition}[theorem]{ทฤษฎีบทประกอบ}
\newtheorem{corollary}[theorem]{บทแทรก}
\newtheorem{observation}{ข้อสังเกต}[]

\theoremstyle{remark}
\newtheorem{example}{Example}
\newtheorem{exercise}{แบบฝึกหัด}
\newtheorem*{remark}{Remark}
% \newtheorem{note}{บันทึก}
\setbeamertemplate{theorems}[numbered]
% \setbeamertemplate{theorems}[ams style] 

\newcommand{\R}{\mathbb{R}}
\newcommand{\N}{\mathbb{N}}
\DeclareMathOperator{\conv}{conv}

\usepackage{animate}
\setbeamertemplate{enumerate items}[default]

\begin{document}
\nocite{*}
\maketitle
\begin{frame}{Outline}
\tableofcontents
\end{frame}
\section{Introduction}
\begin{frame}{Introduction}
\begin{example}
    Find the prime factorization of 2013.
\end{example}
\pause
Because $44<\sqrt{2023}<45$, it is enough to examine the primes $2,3,5,7,11,13,17,19,23,29,31,37,41,43$.\\
We have
\vspace{-1\baselineskip}
    \begin{align*}
        2023&=7\times 289\\
        &=7\times 17\times 17.
    \end{align*}
\end{frame}
\section{Fermat Factorization}
\begin{frame}{Fermat Factorization}
    \begin{theorem}[Fermat Factorization]
        If $n$ is an odd positive integer, then there is a one-to-one correspondence between factorizations of $n$ into two positive integers and differences of two squares that equal $n$. That is,
        $$n=ab=s^2-t^2$$
    \end{theorem}
\end{frame}
\begin{frame}[allowframebreaks,t]{Proof}
    Suppose that $n$ be an odd positive integer with $n=ab$, whenever $a\ge b\ge 1$. Notice that
    \begin{align*}
        n&=ab\\
        &=\left(\dfrac{a+b}{2}+\dfrac{a-b}{2}\right)\left(\dfrac{a+b}{2}-\dfrac{a-b}{2}\right)\\
        &=\left(\dfrac{a+b}{2}\right)^2-\left(\dfrac{a-b}{2}\right)^2
    \end{align*}
    We set $s=\left(\dfrac{a+b}{2}\right)$ and $t=\left(\dfrac{a-b}{2}\right)$ are both integers because $a$ and $b$ are both odd.\vspace{2\baselineskip}\\
    Conversely, assume that $n=s^2-t^2$ where $s,t\in\N$.\\
    Then it is clearly that $n$ can be factored as
    $$n=s^2-t^2=(s-t)(s+t).$$
    Then we choose $a=s+t$ and $b=s-t$.\\
    Moreover, because $n$ is odd integer, then $a$ and $b$\\
    are themselves odd.\hfill$\square$
    % To show that one-to-one correspondence,\\
    % we let a function $f:\{(a,b)\in\N^2\,|\,a,b \text{ be odd}\}\to\N^2$ \\
    % such that 
    % $$f(a,b)=\left(\dfrac{a+b}{2}\right)^2-\left(\dfrac{a-b}{2}\right)^2$$
    % Suppose that $f(a_1,b_1)=f(a_2,b_2)$\\
    % It's clearly to show $a_1=a_2$ and $b_1=b_2$
\end{frame}
\section{The Algorithm}
\begin{frame}{The Algorithm}
To search for possible $x$ and $y$ satisfying the equation 
$$n=s^2-t^2$$
\vspace{-1.5\baselineskip}
    \begin{enumerate}
        \item We write $s^2-n=t^2$.
        \item Determining the smallest integer $k^2\ge n$.
        \item We search for a square among the sequence of integers
        $$k^2-n,\,(k+1)^2-n,\,(k+2)^2-n,\,\ldots$$
    \end{enumerate}
\vspace{-0.5\baselineskip}
\begin{remark}
    It may be necessary to check as many as $\frac{(n+1)}{2}-\lfloor\sqrt{n}\rfloor$ integers to
determine whether they are perfect squares.
\end{remark}
\end{frame}
\section{Example}
\begin{frame}{Example}
    \begin{example}
        Using the Fermat factorization method, factor the 2013.
    \end{example}
    \pause
    We find that $44<\sqrt{2013}<45$, then it suffices to consider values of $k^2-2013$ for those $k$ that satisfy the inequality $45\le k<\dfrac{(2013+1)}{2}=1007$. The calculations begin \\as follows:
    \vspace{-0.5\baselineskip}
    \begin{align*}
        45^2-2013&=2025-2013=12\\
        46^2-2013&=2116-2013=103\\
        47^2-2013&=2209-2013=196=14^2
    \end{align*}\vspace{-1.5\baselineskip}\\
    And then $2013=47^2-14^2=(47+14)(47-14)=61\times 33.$
\end{frame}
\begin{frame}{Example}
\begin{figure}
    \centering
    % \includegraphics[scale=0.3]{qrcode_drive.google.com (1).png}
    \includegraphics[width=0.5\linewidth]{Figure/Final Flash.png}
    \caption{\href{https://docs.google.com/spreadsheets/d/1lInRBG6kQhh8O2SUpPTJ9NZ_3vZ3i2rk4NBUNAxkiGA/edit}{Fermat's Factorization in spreadsheets}}
\end{figure}
\end{frame}
\begin{frame}{References}
\def\newblock{}
\bibliographystyle{plain}
\bibliography{SCMA350Reference}
\end{frame}
\end{document}
